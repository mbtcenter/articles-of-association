% --------------
% LaTeX Document
% --------------

\documentclass[english,a4paper,titlepage,oneside]{article}

% --------
% PREAMBLE
% --------

% Use Packages
% ------------

% Language package
\usepackage[english]{babel}
% Input character encoding
\usepackage[utf8]{inputenc}
% Font encoding package
\usepackage{fontenc}
% Document layout package
\usepackage[top=2.25cm, bottom=3cm, left=2.5cm, right=2.5cm]{geometry}
% PDF document link package
\usepackage{hyperref}

% Document Settings
% -----------------

% Set paragraph vert space and indetatoin.
\setlength\parskip{\medskipamount}
\setlength\parindent{0pt}

% Assert paragraph unit formatting
\raggedright

% Hyphen dissuasion threshold
\hyphenpenalty=10000

% Define interperatation of @ charachter
\makeatletter
\makeatother

% ----------------
% DOCUMENT CONTENT
% ----------------

\begin{document}

% Cover page

\begin{titlepage}
  
  \author{BITCOIN TECHNOLOGY CENTERS, INC.} 
  \title{Articles Of Association} 
  \date{Mon 17 Nov, 2014} 
  \maketitle
  
  \section*{\centerline{Articles of Association of}}
  \section*{\centerline{BITCOIN TECHNOLOGY CENTERS, INC.}}
  \section*{\centerline{Incorporated in Victoria under the}}
  \section*{\centerline{Associations Incorporation Reform Act 2012}}
  The regulations as were in force upon the date of adoption of these Articles shall apply to the Association save to the extent that they are excluded by or are inconsistent with any of these Articles.
  \linebreak[1]
  \\This material is free and unencumbered material is released into the public domain.
  \linebreak[1]
  \\Anyone is free to copy, modify, publish, use, compile, sell, or
  distribute this content, either in source code form or as a compiled
  binary, for any purpose, commercial or non-commercial, and by any
  means.
  \linebreak[1]
  \\In jurisdictions that recognize copyright laws, the author or authors
  of this content dedicate any and all copyright interest in the
  material to the public domain. We make this dedication for the benefit
  of the public at large and to the detriment of our heirs and
  successors. We intend this dedication to be an overt act of
  relinquishment in perpetuity of all present and future rights to this
  content under copyright law.
  \linebreak[1]
  \\THE MATERIAL IS PROVIDED "AS IS", WITHOUT WARRANTY OF ANY KIND,
  EXPRESS OR IMPLIED, INCLUDING BUT NOT LIMITED TO THE WARRANTIES OF
  MERCHANTABILITY, FITNESS FOR A PARTICULAR PURPOSE AND NONINFRINGEMENT.
  IN NO EVENT SHALL THE AUTHORS BE LIABLE FOR ANY CLAIM, DAMAGES OR
  OTHER LIABILITY, WHETHER IN AN ACTION OF CONTRACT, TORT OR OTHERWISE,
  ARISING FROM, OUT OF OR IN CONNECTION WITH THE SOFTWARE OR THE USE OR
  OTHER DEALINGS IN THE CONTENTS.
  \linebreak[1]
  \\For more information, please refer to http://unlicense.org
\end{titlepage}
% Cover page
% Index page
% Content pages
\section*{
  \begin{enumerate}
    \item Interpretation
    \linebreak[1]
    \textnormal{
    \\In these articles:
    \begin{enumerate}
      \item{} ``\emph{the Act}'' means the Associations Incorporation Reform Act 2012 including the Associations Incorporation Reform Regulations 2012 authored by the State in Victoria;
      \item {}``\emph{Address}'' means a postal address or, for the purposes
      of electronic communication, an e-mail address or an addressable 
      communication channel in each case registered with or by the Association;
      \item {}``\emph{the Articles}'' means the Articles of Association
      of the Association;
      \item {}``\emph{the Association}'' means the association intended to be bound by these articles;
      \item {}``\emph{clear days}'' in relation to the period of a notice means
      a period excluding:
      \begin{enumerate}
        \item the day when the notice is given or deemed to be given; and
        \item the day for which it is given or on which it is to take effect;
      \end{enumerate}
      \item {}``\emph{the Committee}'' means the members of the Association so appointed. During any period in which the Association is a charity, the Committee members are trustees as defined by the Charities Act 1997 of Victoria, Australia;
      \item {}``\emph{electronic communication}'' means the same as in the Electronic Communications Act 2000 authored by the State in Victoria;
      \item {}``\emph{Officers}'' includes the Committee and the Secretary;
      \item {}``\emph{the Seal}'' means the common seal of the Association if it
      has one;
      \item {}``\emph{Secretary}'' means the Secretary of the Association or any
      other person appointed to perform the duties of the Secretary of the
      Association, including a joint, assistant or deputy secretary;
      \item {}``\emph{Victoria}'' means the geographic area of Victoria, Australia;
      \item {}``\emph{in writing}'' includes communication via electronic means,
      or by letter; and
      \item words importing one gender shall include all genders, and the singular
      shall include the plural and \emph{vice versa}.
    \end{enumerate}
    } % end textnormal
    \item Identity
    \textnormal{
    \begin{enumerate}
      \item The incorporated association, is to be known as "Bitcoin Technology Centres, Inc." in all holdings.
    \end{enumerate}
    }
    \item Purpose
    \textnormal{
    \begin{enumerate}
      \item The purposes of the incorporated association is to promote and assist the adoption and growth of the Bitcoin trading economy and support businesses and consumers embrace digital currency and related technologies and assist our membership to work on building the future of money.
    \end{enumerate}
    } % end textnormal
    \item Protection of the objects of the association
    \textnormal{
    \begin{enumerate}
      \item No member may be forced against his or her will to accept a change in the objects of the association.
    \end{enumerate}
    } % end textnormal
    \item Membership
    \textnormal{
    \begin{enumerate}
      \item Members may be admitted at any time.
      \item Membership is open to a Natural Persons.
      \item Membership is open to a Legal Entity.
      \item All members have a legal right to resign subject to six months’ notice expiring at the end of the calendar year or, if an administrative period is provided for, at the end of such period.
      \item Membership is neither transferable nor heritable.
    \end{enumerate}
    } % end textnormal
    \item Membership Subscription Fees
    \textnormal{
    \begin{enumerate}
      \item The entrance fees, subscriptions and other amounts (if any) to be paid by members of the incorporated association. If there are no membership fees, you should state that no fees, subscriptions or other payments are required from members. (Item 20 covers the association's source of funds. If no fees are charged, you should not mention fees in your rule addressing item 20.)
      \item Members have a duty to pay subscriptions.
      \item D
    \end{enumerate}
    } % end textnormal
    \item Membership Rights
    \textnormal{
    \begin{enumerate}
      \item The rights, obligations and liabilities of members. This must cover:
      \begin{enumerate}
        \item - which members have the right to vote at general meetings; there must be at least five voting members.
        \item- who will be informed of, and is permitted to attend, general meetings
        \item- who can use facilities of the association
        \item- financial obligations of members.
        \linebreak[1]
        \item Any member who has not consented to a resolution which infringes the law or the articles of association is entitled by law to challenge such resolution in court within one month of learning thereof.
      \end{enumerate}
    \end{enumerate}
    } % end textnormal
    \item Membership Termination
    \textnormal{
    \begin{enumerate}
      \item Provisions for the resignation of a member or cessation of membership. For example, they might have to give written notice to the secretary or to a committee member. They may cease to be a member if their annual subscription is overdue for a substantial period, or they fail to attend a certain number of annual general meetings.
    \end{enumerate}
    } % end textnormal
    \item Dispute Resolution
    \textnormal{
    \begin{enumerate}
      \item If the association lacks one of the prescribed governing bodies, a member or a creditor may apply to the court for an order that the necessary measures be taken.
      \item In particular, the court may set the association a time limit in which to restore the situation required by law and may, if necessary, appoint an administrator.
      \item The association bears the cost of such measures. The court may order the association to make an advance payment to the persons appointed.
      \item For good cause, the association may apply to the court for the removal of the persons it appointed.
    \end{enumerate}
    } % end textnormal
    \item Management Comittee
    \textnormal{
    \begin{enumerate}
      \item The committee is entitled and obliged as defined under the articles of association to manage and represent the association.
      \item The committee shall maintain the association’s business ledgers.
    \end{enumerate}
    } % end textnormal
    \item Management Appointment
    \textnormal{
    \begin{enumerate}
      \item (a) Set out a process for the election or appointment of members of the committee (or governing board) of the association.
    \linebreak[1]
      \item The rules must identify the name by which the governing body is known (this may simply be ?the committee?) and the membership of the committee, such as president, vice president, treasurer or secretary. Such positions are commonly referred to as the office bearers of the association. In addition to office bearers, many associations also have ordinary members on the committee who do not hold a formal title. You may wish to have separate provisions to address the election or appointment of office bearers and ordinary members (if any) to the committee.
    \end{enumerate}
    } % end textnormal
    \item Management Terms
    \textnormal{
    \begin{enumerate}
      \item (b) The terms of office of members of the committee. For example, members may hold office for a term of one year and then be eligible for re-election or reappointment at the next annual general meeting.
      \item (c) Reasons why the office of a member of the committee becomes vacant. Generally these include the end of term, resignation, death, insolvency (bankruptcy) or loss of mental capacity or being removed from office by a resolution of the association at a general meeting.
      \linebreak[1]
      \item A member of the committee of an association vacates office in the circumstances provided in the rules of the association and if the member:
      \linebreak[1]
      \item - resigns by written notice addressed to the committee
      \item - is removed from office by a special resolution
      \item - dies
      \item - becomes insolvent
      \item - becomes a represented person under the Guardianship and Administration Act 1986 (for example, because they suffered an accident that caused a brain injury).
    \end{enumerate}
    } % end textnormal
    \item Interim Management
    \textnormal{
    \begin{enumerate}
      \item (d) The filling of casual vacancies occurring within the committee. This is where a member of the committee ceases to hold office before their term of appointment ends (such as if they resign or die). Your rules could include a process for the committee or president to appoint someone to fill the office until the term for that position ends.
    \end{enumerate}
    } % end textnormal
    \item Meetings
    \textnormal{
    \begin{enumerate}
      \item (e) The quorum and procedure at meetings of the committee. The quorum is the minimum number of committee members who must be present before a committee meeting can be held. You can express this as a whole number or as a percentage or fraction.
      \linebreak[1]
      \item Your rules must also cover the procedure to be followed at committee meetings, such as:
      \linebreak[1]
      \item - how much notice committee members will be given before meetings
      \item - who chairs committee meetings
      \item - order of business
      \item - how decisions are made. Do they need to be unanimous or is there a vote? Does the Chair have a casting vote?
      \item - what should happen if a committee member has a conflict of interest regarding a matter under consideration
      \item - can meetings be conducted using telephone or video communications or does every member have to be physically present in the same place
      \item - who takes minutes of the meeting and what is to be recorded.
    \end{enumerate}
    } % end textnormal
    \item Secretary Cessation
    \textnormal{
    \begin{enumerate}
      \item The procedures for the appointment and removal of the secretary. If the secretary is a member of the committee of your association under your rules, then the procedure for appointment and removal of committee members will be sufficient to cover this.
    \end{enumerate}
    } % end textnormal
    \item Records
    \textnormal{
    \begin{enumerate}
      \item The custody of records, securities and other relevant documents of the incorporated association. Set out who is responsible for this.
    \end{enumerate}
    } % end textnormal
    \item Common Seal
    \textnormal{
    \begin{enumerate}
      \item Provisions for the custody and use of the association’s common seal (if any). Use of a common seal is optional. If your association has one, the rules must state who is responsible for custody of the seal.
      \item Provisions for members to have access to, and to be able to obtain copies of, the records, securities and other relevant documents. These rules should cover:
      \linebreak[1]
      \item - information relating to incorporation, rules, management, membership records and financial statements
      \item - the association’s transactions, dealings, business or property
      \item - which records the committee may refuse to permit inspection of, such as confidential personal, employment, commercial or legal matters
      \item - the right to inspect the register of members, at a reasonable time.
      \linebreak[1]
      \item Under the Act, a member is entitled to inspect the rules of their association and minutes of general meetings of the association at any reasonable time. They are also entitled to a copy of the rules of their association or minutes of general meetings if they make a request in writing to their association for a copy.
      \linebreak[1]
      \item If a member requests to inspect the register of members, the incorporated association must allow this at a reasonable time.
      \linebreak[1]
      \item Your rules may also provide for a member to be able to obtain a copy of the register of members but this is optional – you could provide for inspection but not allow copying.
    \end{enumerate}
    } % end textnormal
    \item Minutes of Meetings
    \textnormal{
    \begin{enumerate}
      \item The preparation and retention (custody and storage) of accurate minutes of (see following two provisions):
      \linebreak[1]
      \begin{enumerate}
        \item (a) general meetings of the incorporated association.
        \linebreak[1]
        \item (b) meetings of the committee or other body having the management of the incorporated association.
      \end{enumerate}
    \end{enumerate}
    } % end textnormal
    \item Transparency \& Disclosure
    \textnormal{
    \begin{enumerate}
      \item The committee shall maintain the association’s business ledgers.
      \item Any person may access and produce copies of minutes of meetings.
      \item Any person may access and produce copies of financial statements.
    \end{enumerate}
    } % end textnormal
    \item Administration
    \textnormal{
    \begin{enumerate}
      \item Right of access (if any) members have to minutes of meetings of the committee, including any terms and conditions subject to which access may be granted. You do not have to allow members access to minutes of committee meetings, but you must state whether or not you will allow it.
    \end{enumerate}
    } % end textnormal
    \item General Meetings
    \textnormal{
    \begin{enumerate}
      \item The general meeting of members is the supreme governing body of the Association.
      \item The general meeting is called by the Committee.
      \item The general meeting of members decides on admission and exclusion of members, appoints the committee and decides all matters which are not reserved to other governing bodies of the association.
      \item It supervises the activities of the Comittee and may at any time dismiss the latter without prejudice to any contractual rights of those dismissed.
      \item The right of dismissal exists by law whenever justified by good cause.
      \item General meetings of the Association will be called to order on the 3rd day of each month from February to December each year.
      \item No General meeting will be held in January each year.
      \item The meeting will come to order at 1815 AEST/DST in Melbourne, Victoria.
      \item All meetings will be conducted by live video-conference and recorded, even where all parties to the meeting are present in person.
      \item Live streams of the meeting shall be broadcast and be open and observable to any person without need of any password or other similar technical exclusion mechanism being applied.
      \item Notwithstanding the above, any group of members representing one-fifth of the membership may direct the Comittee to call a general meeting on any date so requested, provided such a request is made in writting and that the date nominated is not less than 16 clear days of delivering instruction to the Commitee c/o the an Address of the Association.
      \item The comittee shall be deemed to be dissolved where it fails to accede to any such properly formed request of the membership and fails to act to call a general meeting and notify the membership within 7 days of the date nominated. Upon dissolution the membership is responsible to call the meeting to order. The first order of business at a general meeting called by members where the Comittee has been dissolved is to appoint a new Comittee by open ballot. Any member permitted by law to do so must be afforded opportunity to nominate to serve on the comittee prior to the ballot at the meeting. 
    \end{enumerate}
    } % end textnormal
    \item Annual General Meetings
    \textnormal{
    \begin{enumerate}
      \item The Annual General Meeting of the Association shall be held on Hal Finney Day (12th January) each year.
      \item The meeting will come to order at at 1430 in Melbourne, Victoria, Australia.
      \item Annual reports are to be tabled and ratified.
      \item All Officers vacate thier positions at the commencement of the meeting.
      \item The election of a Secretary is the first order of business.
      \item The election of the Comittee is the second order of business.
      \item The outcome of the open ballot is ratified by the membership.
      \item General business (if any) may then be addressed after the new Comittee is ratified.
    \end{enumerate}
    } % end textnormal
    \item Resolutions
    \textnormal{
    \begin{enumerate}
      \item Resolutions are passed by the general meeting.
      \item The written consent of all members to a proposal is equivalent to a resolution of the general meeting.
    \end{enumerate}
    } % end textnormal
    \item Conduct of Meetings
    \textnormal{
    \begin{enumerate}
      \item Set out the procedure to be followed at general meetings. Include:
      \linebreak[1]
      \item All members have equal voting rights at the general meeting.
      \item Resolutions require a majority of the votes of the members present.
      \item Voting is conducted by open ballot with the chair holding a csting vote in the event of a tie.
      \item Resolutions may be taken on matters for which proper notice has not been given only where this is expressly permitted by the articles of association.
      \item Exclusion from voting:
      \begin{enumerate}
        \item Each member is by law excluded from voting on any resolution concerning a transaction or dispute between him or her, his or her spouse or a lineal relative on the one hand and the association on the other.
      \end{enumerate}
      \item - whether and in what circumstances the meeting can be adjourned
      \item - the minimum number or percentage of all members who must be present to conduct a valid general meeting (the quorum)
      \item - whether voting by proxy is allowed.
    \end{enumerate}
    } % end textnormal
    \item Notice of Meetings
    \textnormal{
    \begin{enumerate}
      \item Set out the period of notice required to be given to members of a forthcoming general meeting and the manner in which notice is to be given (such as post or email or both). The rules must also set out the period of notice (if any) a member must give other members if they propose to move a motion at a general meeting. If you wish to allow for members to propose motions from the floor at a general meeting, state that no advance notice is required.
      \linebreak[1]
      \item The Act imposes specific requirements for notice where the motion requires a special resolution for it to be passed.
      \linebreak[1]
      \item A motion proposing an alteration to the rules of an incorporated association must be passed by a special resolution.
      \linebreak[1]
      \item A special resolution must be passed by at least 75\% of the members present or voting by proxy at a general meeting
      \linebreak[1]
      \item Members must be given at least 21 days’ notice of a motion that is to be passed by special resolution. The notice must include:
      \linebreak[1]
      \item - the date, time and place of the meeting
      \item - the full proposed resolution
      \item - a statement of the intention that the motion be proposed as a special resolution.
    \end{enumerate}
    } % end textnormal
    \item Financial Resources
    \textnormal{
    \begin{enumerate}
      \item The association is liable for its obligations with its assets. Such liability is limited to the assets.
      \item The sources from which funds of the incorporated association will be or may be derived. These may include fees paid by members, grants and donations, proceeds from the sale of products or materials (if relevant).
    \end{enumerate}
    } % end textnormal
    \item Financial Management
    \textnormal{
    \begin{enumerate}
      \item The manner in which the funds of the incorporated association must be managed and, in particular, the mode of drawing and signing cheques on behalf of the incorporated association. Include who:
      \linebreak[1]
      \item - is responsible for receiving funds on behalf of the association and issuing receipts for those funds
      \item - is responsible for paying funds received into the associations’ bank account
      \item - can authorise expenditure by the association and how authorisation is given (such as by signature of the treasurer and another committee member)
      \item - can sign cheques on behalf of the association and what authorisation they need to do so (such as resolution of the committee).
    \end{enumerate}
    } % end textnormal
    \item Altering These Articles
    \textnormal{
    \begin{enumerate}
      \item Provide for the process to be followed to alter your rules, including adding new rules and removing old rules. This must be consistent with the requirements of the Act and a simple statement that the rules of the association may be altered by special resolution at a general meeting of the association is sufficient.
    \end{enumerate}
    } % end textnormal
    \item Winding Up and Dissolution
    \textnormal{
    \begin{enumerate}
      \item The association may be dissolved at any time by resolution of the members.
      \item The association is dissolved by operation of law if it is insolvent or if the committee may no longer be appointed in accordance with the articles of association.
      \item By court order where the objects of the association are unlawful or immoral, the competent authority or an interested party may apply for a court order of dissolution.
      \item Set out what is to happen to any surplus assets of the association if it is wound up or dissolved. The distribution of surplus assets must not be contrary to the Act and generally surplus assets must not be distributed to any member or former member of the association.
    \end{enumerate}
    } % end textnormal
  \end{enumerate}
  }
\end{document}